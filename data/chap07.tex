\chapter{总结与展望}
本章对本文的研究进行总结,针对课题研究中待完善的地方,提出下一步工作的设想。
\section{本文总结}
无线传感网在许多大范围监测领域都有广泛的应用,在环境监测、军事侦察等领域都有规模化的应用。无线传感器节点通常被部署在
环境恶劣的无人值守区域,容易节点受损或者节点受攻击,导致传感网的传输受到严重影响。因此需要对无线传感网中的传输进行认证,保证数据传输的安全性。但是由于无线传感器节点计算能力、存储空间有限,而且节点能量受到限制,因此安全机制的设计必须满足轻量级的要求,以适应传感器节点的限制。我们设计了多跳长路径上多节点联合的数据认证机制,有效利用了数据认证机制保障了传输路径上的安全。

本文主要工作总结如下:
\begin{enumerate}\setlength{\itemsep}{-\itemsep}
  \item 深入研究了无线传感网中数据认证及其密钥分配的理论,分析了它们的发展研究现状,分析了数据认证在无线传感网中的应用,并总结了各种方案的优点和宝贵经验。
  \item 提出了多跳长路径上多节点联合数据认证的模型,设计实现了多跳长路径上多节
        点联合数据认证协议,并设计了路径上节点关系的维护算法,对协议的安全
        性能进行了分析评价。
  \item 针对多跳长路径上多写点联合数据认证协议的不足,对算法进行了优化,提
        出了多路径抗节点失效机制和动态步长多节点联合数据认证机制,并对优化
        方案的安全性能进行了分析评价。
  \item 围绕多跳长路径多节点联合数据认证机制的需求,对密钥分配方案进行了深
        入研究,提出了基于单向 hash 链的密钥分配方案,并对认证中的 MAC 进行
        了研究,提出了适应数据认证机制需求的 MAC 码。
  \item 通过仿真实验验证了本文中提出的方案的安全性能以及优化效果。
\end{enumerate}



\section{未来工作于展望}
本文设计实现的多跳长路径上多节点联合的数据认证机制能够很好地提升了无线传感网数据传输的安全性,取得了一定的成果。但是由于无线传感网本身的复杂性以及部署场景的多样性,我们的机制对不同工作场景下的无线传感网的适应性还需进一步研究,在数据认证的具体算法以及密钥分配等方向还有进一步研究的空间。结合现阶段以及完成的工作,我们未来可做的相关工作有:

\begin{enumerate}\setlength{\itemsep}{-\itemsep}
  \item 本文研究的数据认证机制与无线传感网现有的各层次协议结合还不够,在后续的工作中可以将数据认证机制同其他协议结合起来,例如同网络层的路由协议相结合,将数据认证的相关报文使用路由报文发送。
  \item 针对本文的数据认证机制提出的两个优化方案,虽然在仿真实验环境中已经进行了验证,但在实际的无线传感网中部署比较复杂,需要通过搭建真实的传感器节点平台,对相关方案进行针对性的优化。
  \item 密钥分配方案的设计实现还不够完善,需要对相关的算法进行进一步的改进,形成完整的密钥管理机制,并对密钥分配方案的效率和安全性进行分析评价。使用的单向hash函数还有改进的空间,可以进一步压缩其计算开销。
\end{enumerate}





