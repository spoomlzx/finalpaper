\chapter{仿真实验与结果分析}
在第三章中我们设计实现了多跳长路径上多节点联合数据认证机制,第三章中提出了对多节点联合数据机制的优化方案,并对它们进行了理论分析。在本章中我们将使用仿真实验平台,对数据认证机制进行实验验证,并对它们的性能进行比较分析。6.1节介绍实验平台的环境,及相关仿真软件的介绍。6.2节是多跳长路径上多节点联合数据认证仿真实验及结果分析。6.3节是数据认证优化方案仿真实验及结果分析。
\section{实验平台环境}
\subsection{仿真实验环境}
为了搭建无线传感网的仿真环境,本文采用了仿真平台OMNeT++(Objective Modular Network testbed in C++)进行仿真,在Windows 7 的主机中安装的版本为OMNeT++4.6b1。 主机的配置为:
\begin{compactitem}
  \item DDR3内存,容量8G,频率1600MHz
  \item Core 双核处理器,主频2.6GHz
\end{compactitem}

现在进行网络仿真的工具很多,还有有NS2、NS3、OPNET、TOSSIM、GloMoSim和PowerTOSSIM等平台。NS2和NS3平台虽然有丰富的协议库支持,但是仿真时数据量过大,对复杂的无线传感网场景的仿真不合适,而且学习曲线比较高。OPNET平台要实现无线传感网仿真,需要添加能量模型,而且其最大问题是仿真速度慢,效率会随网络规模和流量增大而降低。TOSSIM、GloMoSim和PowerTOSSIM等其他仿真平台也有相应的局限性。

本文的实验中,我们选择了OMNeT++ 平台,OMNeT++是面向对象的离散事件仿真工具。OMNeT++基于Eclipse平台构建,是一款开源的仿真集成环境,由建模工具、仿真运行工具和输出结果分析工具等组成。OMNeT++还有INET、INETMANET或MIXIM等类库支持网络仿真,利用这些开源的包,很容易对各种网络协议进行仿真实验。

在OMNeT++中,主要使用C++语言创建仿真对象类,并使用ned文件配置仿真的参数,一个OMNeT++的仿真模型包括如下部分:
\begin{enumerate}\setlength{\itemsep}{-\itemsep}
  \item 网络拓扑结构描述文件(.ned文件):使用ned文件对仿真网络中的信道、门、连接和模块等进行配置。
  \item 消息或包定义文件(.msg文件):对网络中传输的消息或者数据报文进行定义,OMNeT++会自动从.msg生成相应的消息类。
  \item 模块描述文件(.h和.cc文件):使用C++语言描述模型中的各个模块。
\end{enumerate}
仿真模型使用C++编译器将.ned网络拓扑结构描述文件、.msg消息或包定义文件以及模块描述文件进行编译后,同仿真内核库以及用户接口库进行链接,生成可执行的仿真程序。OMNeT++不同于其他仿真平台,生成的仿真程序可以运行于没有安装OMNeT++的机器上。

\subsection{无线传感网数据认证仿真框架}
下面将对搭建的无线传感网数据认证仿真模型进行介绍,
在我们的无线传感网数据认证机制研究中,我们的仿真模型主要包括两种对象:传感器节点和基站。
在本文中,我们假设所有的传感器节点均为相同的配置,部署时的能量相同。

\subsubsection{模块配置}
在OMNeT++仿真平台中,使用ned文件配置相应的模块,
传感器节点的配置文件如下所示,节点包括了一个输入输出门向量:
\begin{lstlisting}[language=C]
simple Node
{
    parameters:
        double txRate @unit(bps);
        double radioDelay @unit(s);       // 时间片
        volatile double iaTime @unit(s);  // 数据包发送间隔时间
        @display("i=device/wsn");
    gates:
        inout out;
}
\end{lstlisting}
基站的配置文件如下所示,基站也包括一个输入输出门向量,仿真实验中假设基站的通信不会受信道冲突的影响:
\begin{lstlisting}[language=C]
simple Bs
{
    parameters:
        @display("i=device/antennatower_l");
    gates:
        inout bsRadio[];
}
\end{lstlisting}
\subsubsection{模块类实现}
在我们的仿真实验中,传感器节点模块和基站模块都使用C++编写,我们对两个模块的实现做简要的介绍。

传感器节点使用了固定时间间隔内随机生成数据报文的方式仿真数据采集的过程,并采用自消息完成数据传输给簇头的过程,其采集过程函数表示如下:
\begin{lstlisting}[language=C]
void Node::initialize() {
    char msgname[20];
    sprintf(msgname, "event sensor # %d", getIndex());
    startEvent = new cMessage(msgname);
    iaTime = &par("iaTime");
    radioDelay = par("radioDelay");
    //事件捕获后的发送时间
    simtime_t t = simTime() + iaTime->doubleValue();
    if (ev.isGUI())
        getDisplayString().setTagArg("t", 2, "#808000");
    scheduleAt(t, startEvent);
}
\end{lstlisting}

传感器节点仿真发送数据包的过程主要由函数handleMessage完成:
\begin{lstlisting}[language=C]
void Node::handleMessage(cMessage* msg) {
    int src = getIndex();
    char msgname[40];
    sprintf(msgname, "event from cluster node %d", src);
    PacketCn *pkt = new PacketCn(msgname);
    pkt->setBitLength(952);
    pkt->setSource(src);
    pkt->setEvent(event);
    WATCH(pkt);
    //下一个事件发送时间,形成一个包传送给BS
    simtime_t tt = simTime() + iaTime->doubleValue();
    simtime_t nextTime=0;
    if((ceil(simTime()/radioDelay)*radioDelay)<tt)
        nextTime = tt;
    else
        nextTime = tt+radioDelay;
    ev << "send packet:"<<pkt<<" when "<<nextTime<< endl;
    send(pkt, "out$o");
    scheduleAt(nextTime, startEvent);
}
\end{lstlisting}

基站模块主要是接收数据报文,并对报文中的XMAC进行认证,确定报文的身份,其主要函数如下:
\begin{lstlisting}[language=C]
void Bs::handleMessage(cMessage* msg) {
    PacketCh *pkt = check_and_cast<PacketCh *>(msg);
    int event=pkt->getEvent();
    //计算event的XMAC
    int eventMac=this->Xmac(event);
    if(pkt->getXmac()==eventMac){
        ev<<"event received!"<<endl;
    }
    else{
        ev<<"trash packet!"<<endl;
    }
    delete pkt;
}
\end{lstlisting}
\section{多跳长路径上多节点联合数据认证仿真实验}
在



\section{数据认证优化方案仿真实验}





\section{本章小结}
本章首先介绍了仿真实验环境的搭建,对使用OMNET++进行无线传感网的仿真进行了介绍,对相关实验的框架搭建进行了说明。使用OMNET++ 平台对我们的多跳长路径上多节点联合的数据认证机制进行仿真实验,验证该数据认证方案有较好的安全性能。
我们还对两个优化方案进行了仿真,实验结果表明两个方案分别对检测成功率和通信开销有较好的改进。
