\chapter{绪论}
本章首先对本文研究的课题背景和研究意义进行介绍,然后对本文的研究内容和文章的
组织结构予以说明。
\section{本文研究背景和意义}
规模化传感网既是前沿研究问题,又在国家社会重大安全领域有着预期的应用前景,预期将得到广泛应用和持续的发展。本研究适应大规模大数据无线传感需求的数据认证机制,设计相关认证算法、协议及有关密钥分配管理的算法,并将在仿真平台和实际传感器节点平台上实现、验证和测试有关功能。将为大规模大数据无线传感网的数据认证提供可行的技术方案,为了解决无线传感网的安全传输提供了工程上的解决范例。



\subsection{无线传感网概述}
正文内容

正文内容


\subsection{无线传感网对数据认证的需求}
无线传感网的安全需求主要有两方面:面向基础设施的通信安全及面向数据的信
息安全。前者为无线传感网完成数据采集、数据传输、数据融合等基本功能提供支
撑,后者提供实现数据机密性、完整性、不可否认性等信息安全机制。
通信安全主要包括以下内容:
- 节点安全性:指传感器节点的部署隐蔽性及抗受损能力。要求节点不易被发现,
且节点内部通过代码混淆等方法提供一定的机密信息保护措施。
- 防御能力:指无线传感网抗外部攻击及内部攻击的能力。要求在敌手攻击下,部
分节点受损不会影响网络的整体功能。
- 入侵检测能力:指识别入侵行为,确定入侵者身份、位置等信息,主动丢弃入侵
者发出的虚假数据的能力。

信息安全主要包括以下内容:
- 数据机密性:通过加密技术及访问控制能确保网络中的消息明文不会暴露给非授
权实体。
- 不可否认性:通过消息签名、身份认证、访问控制等方式有效识别消息源。
- 数据完整性:通过消息认证码(MAC)、消息签名等技术提供数据在传输过程中的
防篡改能力。
- 数据新鲜性:通过消息新鲜性参数确保消息在其时效范围内被目标实体接收。


除了保证必要的通信安全及数据安全,安全协议的设计还应当考虑如下因素:
- 抗毁性:部分节点受损不会导致整个网络安全体系瘫痪。
- 可扩展性:安全协议不应当对网络节点的加入与移除造成影响。
- 灵活性:安全协议不应当影响网络部署的灵活性。
- 低开销:安全协议带来的计算开销、通信开销、存储开销及对应能耗应当是传感
器节点可承受的。

\section{本文研究内容}
正文内容

正文内容

正文内容

正文内容

\section{本文组织结构}


